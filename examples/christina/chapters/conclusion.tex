%======================================================================
\chapter{Conclusion} \label{chapter:conclusions}
%======================================================================

This work focused on several fundamental problems of genomic sequence analysis: motif recognition, the {\sc Closest String} problem, and other distinguishing string problems. We constructed predictive models for tasks in pattern recognition, and identified and applied many combinatorial and probabilistic insights to our problems of interest.  Modelling biological problems as graphs and other abstract mathematical objects can lead to theoretical results concerning the computational complexity and thus, the ability to find an approximate solution efficiency.  From a practical perspective, this area of research can lead to powerful new tools for identifying genes and possible mutations. 

We described several original contributions -- both theoretical and applied.  Our main contributions are summarized below:

\begin{itemize}
\item We developed a new approach for motif recognition, and provided theoretical and experimental results that support our novel model and algorithm \cite{boucher07}. Our algorithm, MCL-WMR, builds a weighted graph model of the input data and uses a graph clustering algorithm to quickly determine important subgraphs that need to be searched further for valid motifs. Our experimental results show that MCL-WMR has competitive running time capabilities and accuracy.  
\item We gave a linear-time algorithm for solving {\sc Closest String} instances with a small number of strings; which addressed an open problem of Gramm {\em et al.}\ \cite{GNR03}.  We also considered the dual problem -- instances with a large number of strings -- and provided empirical results that demonstrate that these ``large'' instances can be solved efficiently.  Our analytical explanation, as to why {\sc Closest String} instances with a large number of strings are easily solved in practice, involved initiating the study of the smoothed complexity of the {\sc Closest String} problem.  
\item We proposed a refined closest string model, the {\sc Closest String with Outliers} (CSWO) problem, which asks for a center string $s$ that is within Hamming distance $d$ to at least $n-k$ of the $n$ input strings, where $k$ is a parameter describing the maximum number of outliers. A CSWO solution not only provides the center string as a representative for the set of strings but also reveals the outliers of the set.  We gave fixed parameter algorithms for CSWO when $d$ and $k$ are parameters, for both bounded and unbounded alphabets. We also proved that when the alphabet is unbounded the problem is W[1]-hard with respect to $n-k$, $\ell$, and $d$.
\item  Lastly, we applied the probabilistic heuristics and combinatorial insights for the {\sc Closest String} problem to motif recognition.  This program, referred to as {\em sMCL-WMR}, is used to uncover similarities in the promoter region of the genomic data of canola.  We identified more than 40 motifs in the three promoters that might be responsible for certain biological activities concerning the seed coat, and synthetically developed a promoter that is conjectured to express all biological activities of interest. This synthetic promoter DNA sequence  is currently being introduced into canola and tested for its expression.
\end{itemize}

Throughout this thesis we have suggested open problems that warrant further investigation.   We conclude by giving further details on some of these suggested open problems, as well as proposing some future research directions.

\subsubsection{Solving Small Motif Instances}

In Chapter \ref{chapter:closest_string_problem} we described a simple, linear-time algorithm for solving the {\sc Closest String} problem.  Since the development of this algorithm, these results have been extended by Amir {\em et al.}\ \cite{amir} to a variant of the optimization version of the {\sc Closest String} problem that minimizes both the maximum Hamming distance but also the sum of (Hamming) distances from the strings to the center string. It is an open problem as to if there exists an efficient, polynomial-time algorithm for the {\sc Closest String} problem restricted to four strings and an alphabet larger than the binary one, or for the {\sc Closest String} problem restricted to a constant number of strings greater than four.  In addition, extending the results of Amir {\em et al.}\ \cite{amir} to larger alphabets or sets of strings is currently open.

\subsubsection{Smoothed Analysis}

Numerous open problems still remain in the area of studying smoothed analysis and distinguishing string selection problems. Studying how robust the best (or good) instances rather than how fragile worst-case instances are warrants further consideration, as well as studying the smoothed analysis of the PTAS of Li {\em et al.}\ \cite{LMW02}.  The smoothed complexity of other distinguishing string selection problems remains open, these include the {\sc Farthest String} problem and other variants of the {\sc Closest String} problem.  Reconciling the complexity of these problems would further conclude about the ability to solve practical instances of variants to the {\sc Closest String} problem.

\subsubsection{Practical Algorithmic Solutions for Solving CSWO}

We gave several fixed parameter tractability algorithms for {\sc CSWO} in Chapter \ref{chapter:cswo}.  The practicality of these algorithms has not been investigated, and the worst-case analysis implies that they may be impractical for solving relatively large instances of this problem.  Since there exist practical applications to this problem, it is imperative that algorithms for {\sc CSWO} that are efficient and accurate in practice be developed.

\subsubsection{Sampling and Counting Center Strings}

In Chapter \ref{chapter:pmclwmr} we showed the applicability of sampling and counting center strings to motif recognition.  There presently remain many open problems concerning this topic. Boucher and Omar gave results concerning the computational difficulty of sampling and counting center strings \cite{boucher_omar}. Developing efficient methods to sample and count center strings presently remains open.  The development of a rapidly mixing {\em Markov chain Monte Carlo (MCMC)} algorithm could show the existence of an algorithm for uniform at random generation of pairwise bounded sets and should be further explored.  MCMC methods have been successful in producing a sampling method for some combinatorial sampling problems. Extending Barvinok's algorithm \cite{barvinok}, which counts the number of points in a polytope, to a weighted polytope would also give the desired result by combining it with the integer linear programming formulation of Boucher and Omar \cite{boucher_omar}. 