% \section{Conclusion}

%Previously, we demonstrated that the succinct colored de Bruijn graph could represent and find variants in large populations given enough time and space to construct the data structure ~\cite{vari}.  However, this can still be considerable, taking almost 90 hours in the most extreme case of our previous experiments.  The Achilles heal of this succinct colored de Bruijn graph is that all the source data must exist simultaneously in some form of memory (i.e. it cannot be streamed)  and must be manipulated in its uncompressed form in order to generate the compressed form.  Moreover, while $\vari$ performs this in external memory and thus, ensures that amount of RAM does not need to exist, even external memory can be limited and is considerably slower than RAM.  This fact encumbers current and future use of this approach with the largest datasets.  

In this paper, we propose to further increase the scalability of succinct colored de Bruijn graphs by developing a method to merging smaller graphs in a resource-efficient manner.  This allows the colored de Bruijn graph to be constructed for massive size datasets.  In addition, our algorithm provides an efficient means to update a succinct colored de Bruijn graph with additional data  as it becomes available. This is useful for example, in the GenomeTrakr database, which is continually being updated with more data on a monthly (or even weekily basis) and the search for a foodborne outbreak requires the analysis of the complete dataset.  Thus, rather than rebuilding the colored de Bruijn graph on the new (complete) version of the GenomeTrakr data, dynamically updating the existing one  would ensure ideal use of time and resources.  

Lastly, our merge algorithm may be applicable to  to other prefix-only compressed suffix arrays such as GCSA by Sir{\'e}n {\it et al.}~\cite{siren2014indexing} and XBW by Ferragina {\it et al.}\cite{ferragina2009compressing}.  This merits future investigation as these data structures are of both theoretical and practical interest. 
