\section{Discussion and Conclusions}
\label{sec-discussion}

We demonstrated that $\twin$, an index-based algorithm for aligning {\em in silico} digested contigs to an optical map, gave over an order of magnitude improvement to runtime without sacrificing alignment quality. Our results show that we are able to handle genomes at least as large as the budgerigar genome directly, whereas SOMA cannot feasibly complete the alignment for this genome in a reasonable amount of time without significant parallelization, and even then is orders of magnitude slower than $\twin$. Indeed, given its performance on the budgerigar genome, and its $O(m^2 n^2)$ time complexity, larger genomes seem beyond SOMA.  For example, the loblolly pine tree genome, which is approximately 20 Gb \cite{pinetree}, would take SOMA approximately 84 machine years, which, even with parallelization, is prohibitively long.
%is theoretically possible but would required an enormous amount of parallelization and CPUs.

Lastly, optical mapping is a relatively new technology, and thus, with so few algorithms available for working with this data, we feel there remains good opportunities for developing more efficient and flexible methods. Dynamic programming optical map alignment approaches are still important today, as the assembly of the consensus optical maps from the individually imaged molecules often has to deal with missing or spurious restriction sites in the single molecule maps when enzymes fail to digest a recognition sequence or the molecule breaks.  Though coverage is high (e.g. about 1,241 Gb of optical data was collected for the 2.66 Gb goat genome), there may be cases where missing restriction site errors are not resolved by the assembly process.   In these rare cases (only 1\% of alignments reported by SOMA on parrot contain such errors) they will inhibit $\twin$'s ability to find correct alignments.  In essence, $\twin$ is trading a small degree of sensitivity for a huge speed increase, just as other index based aligners have done for sequence data.  Sir\'{e}n et al.~\cite{dag_method} recently extended the Burrows-Wheeler transform (BWT) from strings to acyclic directed labeled graphs and to support path queries. In future work, an adaptation of this method for optical map alignment may allow for the efficient handling of missing or spurious restriction sites.

%I'm not sure if it's punchy, but if all you want to do is know how many approximate matches exist, we can skip the step of converting BWT intervals to original "text" intervals.  There is also the fact that we match all approximate patterns concurrently.  This is the same argument as in BWA which says "Because exact repeats are collapsed on one path on the prefix trie, we do not need to align reads against each copy of the repeat."  In our case, it's not just DNA sequence repeats, but since ORM data has less resolution, there may be multiple non repeat DNA strings that happen to have the restriction enzyme target motif in the same place.
